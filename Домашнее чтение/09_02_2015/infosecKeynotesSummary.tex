\documentclass[14pt]{article}

\title{InfoSec2015 Keynotes Summary}
\date{\today}
\author{Igor Pevtsov, 53501/3}
 
\begin{document}
  	\maketitle

	In modern world status of Computer Science is ambiguos, it belongs to several Colleges simultaneously. The large-scale adoption of internet services across diverse global populations is one indicator that the average consumer is reasonably comfortable with the collateral risks. US Department of Defence recognized cyberspace as a man-made domain within which wars will be conducted. 20-30 years before, desicion makers did not  see the Internet as something important, but today the situation is different: they realize how it can be useful in doing surveillance on citizens. The US NSA program called PRISM is one of the most powerful instruments for surveillance and monitoring people all over the world. Also, USA has the exceptional opportunities for monitoring hundreds millions of people because all of the most popular internet services are located in the USA (Google, Facebook, Twitter, etc.).Edward Snowden example shows that information leaking is possible everywhere, even in the Intelligence Agencies.
	 The success of the Internet has created a dependency on communication as many of the processes underpinning the foundations of modern society would grind to a halt should communication become unavailable. Recent patches to improve Internet security and availability have been constrained by the current Internet architecture, business models, and legal aspects. Important challenge is how to scale authentication of entities (e.g., AS ownership for routing, name servers for DNS, or domains for TLS) to a global environment. To solve all this problems, the design of a next-generation Internet, that is secure, available, and offers privacy by design; that provides appropriate incentives for a transition to the new architecture; and that considers economic and policy issues at the design stage, is being studied.
\end{document}
