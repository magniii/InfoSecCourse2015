\documentclass[14pt]{article}

\title{InfoSec2015 Summary 3}
\date{\today}
\author{Igor Pevtsov, 53501/3}
 
\begin{document}
  	\maketitle

	\section{Detecting Roles and Anomalies in Hospital Access Audit Logs (p1-gunter) summary}
	\begin{itemize}
	\item There is significant risk in denying access to a healthcare provider who seeks to examine a patient’s medical record. 
	\item Hospital audit logs are too large to audit manually so models and automation are needed to support accountability. 
	\end{itemize}

	\section{Improving Efficiency of Spam Detection using Economic Model (p11-gillani) summary}
	\begin{itemize}
	\item Economic lifting has made email spam a scathing threat to the society due to its related exploits. Many spam detection schemes have been proposed employing the tendency of spam to alter the normal statistical behavior of mail traffic. 
	\item  Different detectors exploits different statistical features to segregate spam from benign (normal). Nevertheless, threshold tuning is still a challenging task as, shooting down benign emails as spam, in a pursuit of higher detection rates, can be detrimental.
	\item Researchers assigned static IPs to 6 machines and kept them unprotected for 24 hours, and within this time they were infected by many malwares. The firewall was used to stop malware spreading and to route SMTP packets to a sinkhole mail server where it was captured and written to a file. The mail server was set to respond any SMTP requests from bots to fool them as being connected to relay servers. These machines were monitored for two months and they have collected approximately 3 million spam emails.
	\item Researchers used such features as Inter-Departure Time, Emails per Recipient, Email Size and Distribution of New Recipients in their evaluation.
	\item In order to maximize the use of commodity, the spammer would want to use 0 second IDT to send as many spam emails as possible. 
	\item A spammer would want to send only a reasonable emails to any recipient to raise any suspicion.
	\item The spammer must have to vary the size of the emails to stay close to the benign behavior.
	\item It costs around dollars 10−50 for a spammer to send one million spam emails. Researchers used value of 20 dollars.
	\item A spammer needs to send approximately 10 millions emails to get one positive response.
	\item The spam detectors suffer from the inherent tradeoff between accuracy and efficiency. 
	\end{itemize}

	\section{Enterprise Risk Assessment Based on Compliance Reports and Vulnerability Scoring Systems (p25-alsaleh) summary}
	\begin{itemize}
	\item The risk of cyberattacks have become increasingly daunting as most of our socioeconomic activities have gone cyberbased. Comprehensive automated risk management is becoming necessity in today’s dynamic networks.
	\item Risk assessment is a key component in Risk management. It is not realistic to assume any computing system completely secure as every system has vulnerabilities of one kind or another (software flaws, misconfigurations, etc.).
	\item The Security Content Automation Protocol (SCAP) provides standard specifications to communicate security information. The Extensible Configuration Checklist Description Format (XCCDF), a SCAP specification, specifies a language to describe security checklists and collect compliance results of targeted systems. SCAP also provides a set of measurement and scoring systems to provide universal scores of known vulnerabilities. SCAP-compliant tools usually scan the targeted systems as stand alone entities.
	\item The integration of network configuration provides live information about the countermeasures and assets distribution in the network, which leads to precise exposure and risk estimation. 
	\item The complete configuration of a network includes the network topology along with the policies of network devices. The network devices can be roughly classified as hosts and middle-boxes (i.e. routers, firewalls, intrusion detection systems, etc.). Each middle-box has its own policy format and structure. 
	\item The possible isolation levels differ from one network to another. Each isolation level is associated with a value that intuitively determines the ability of counteracting attacks. 
	\item NIST Interagency Report 7502 categorizes the vulnerabilities into three categories: (1) software flaw vulnerabilities , (2) security configuration issue vulnerabilities, and (3) software feature misuse vulnerabilities.
	\item The Infidelity is a quantitative value that measures the likelihood of a particular host to act as a threat source.
	\item The risk associated with a host is proportional to its exposure, its asset value, and the number and severity of its vulnerabilities.
	\item This work investigates the feasibility of using security compliance reports along with universal vulnerability scores and network configuration in assessing the risk of cyberattacks. 
	\end{itemize}
	
	\section{Protecting Enterprise Networks through Attack Surface Expansion (p29-sun) summary}
	\begin{itemize}
	\item Attack surface is a valuable metric that helps administrators of enterprise networks to evaluate the risk and security of the entire network. In contrast to the traditional attack surface reduction, researchers propose a new attack surface expansion (ASE) mechanism that focuses on expanding the external attack surface, so that external attackers cannot easily identify the real internal attack surface from a much larger external attack surface.
	\item It is critical to protect the enterprise assets from being stolen or compromised by internal and external attackers. 
	\item A system’s attack surface is the set of ways in which an adversary can enter the system and potentially cause damage.
	\item The external attack surface is the vulnerabilities collected by external attackers through network reconnaissance. Since the network administrator has a direct access on most network components, it can discover a larger attack surface through internal scanning and testing. Therefore, the internal attack surface is usually larger than the external attack surface. Attack surface reduction (ASR) closes all but required doors leading to system assets and constrains others with access rights, monitoring, and response. 
	\item The goal of attack surface expansion is to enlarge the external attack surface that can be observed by the external attackers.  The administrator expands the external attack surface so that the attackers cannot locate the real system and misuse its vulnerabilities.
	\item ASE Techniques: Expanding AttackSurface using Virtual Identities, Expanding Attack Surface using Secret Moving Proxy, Expanding Attack Surface using Dynamic Virtual Networks.
	\end{itemize}
\end{document}
